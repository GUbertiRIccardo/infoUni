\documentclass[a4paper, 12pt]{article} % ogni documento deve contenere almeno la dichiarazione della classe. 


%\usepackage[T1]{fontenc}
\usepackage[italian]{babel}
\usepackage[utf8x]{inputenc}
\usepackage{lipsum} % per generare testi lunghi a caso
\usepackage{multicol} % per impaginare con più colonne
\usepackage{enumerate} % gestisce elenchi puntati
\usepackage{enumitem} % gestisce elenchi puntati
\usepackage{verbatim} % per scrivere pezzi di codice interpretati letteralmente

%%%% MATEMATICA
\usepackage{amsmath}
\usepackage{amssymb}

%%%% FIGURE
\usepackage{graphicx}
\usepackage{caption}
% \captionsetup{tableposition=top, figureposition=bottom} % solo specificando top in questo modo le caption scritte sopra alla figura verranno visualizzate sopra alla figura.

%%% BIBLIOGRAFIA
% specifichiamo lo stile delle citazioni
\bibliographystyle{plain}

%%%% nuovi comandi definiti dall'utente
\newcommand{\ta}{\left(} % parentesi tonda aperta
\newcommand{\tc}{\left)} % parentesi tonda chiusa
\newcommand{\ecm}[2][kJ]{\ensuremath{E_{\textrm{CM}} = {#2}}~\textrm{#1}} % struttura più complessa. Si assicura che sia scritto in ambiente matematico ed elimina la formattazione matematica in parte di essa (textrm). Inoltre prende in input due parametri e specifica il default per il primo di essi: "kJ"


\title{Il mio template perfetto per \LaTeX} % \LaTeX è un comando speciale che stampa il simbolo logo di latex

\date{\today} % per non stampare la data usare \date{}
\author{Nicolò Valle, nicolo.valle@unipv.it}

%%%%%%%%%%%%%%%%%%%%%%%%%%%%%%%%%%%%%%%%%%%%%%%%%%%%%%%%%%%
% qui finisce il PREAMBOLO
%%%%%%%%%%%%%%%%%%%%%%%%%%%%%%%%%%%%%%%%%%%%%%%%%%%%%%%%%%%
\begin{document} 

\maketitle  % stampa titolo, autore e data

\begin{abstract}
  Questo è l'abstract del mio articolo.
  \lipsum[1] % un testo abbastanza lungo solo per controllare l'impaginazione
\end{abstract}


\tableofcontents % stampa l'indice: compilare due volte per aggiornarlo


% \chapter{Capitolo con Hello world} % i capitoli non si possono usare con article

\section{Hello world}

% Gli spazi consecutivi sono ignorati. Solo uno spazio viene stampato
Hello world! Questo                  documento è stato generato il \today.

%\vspace{1cm} % per forzare una spaziatura verticale

E oggi è proprio quel giorno.
Anche andare a capo una volta viene interpretato come un singolo spazio.

Occorre lasciare una linea bianca per andare davvero a capo.

% \twocolumn % due colonne da questo punto in poi. Si potrebbe anche usare l'ambiente \begin{multicols}{2} .... \end{multicols}




\section{Elenchi puntati}

\begin{itemize} % enumerate per gli elenchi numerati
\item Prima linea
\item[-] Seconda linea % ho cambiato lo stile del bullet
\end{itemize}


\section{Matematica}

% un'equazione numerata : \begin{equation}... \end{equation}
\begin{equation}
  \label{eq:sin} % in qualunque punto del testo potrò chiamare \ref{eq:sin} 
  g(x;\mu,\sigma) = \frac{1}{\sqrt{2\pi}\, \sigma} \, e^{-\frac{1}{2} \frac{(x-\mu)^2}{\sigma^2}} \quad \textrm{and} \quad \int_{\mathbb R} g(x;\mu,\sigma) dx = 1 \ \forall \ \mu, \, \sigma > 0 
\end{equation}

% come sopra, ma il nome  è personalizzato tramite il comando \tag{}
\begin{equation}
  \label{eq:sin2} % evitare di ridefinire lo stesso nome
  g(x;\mu,\sigma) = \frac{1}{\sqrt{2\pi}\, \sigma} \, e^{-\frac{1}{2} \frac{(x-\mu)^2}{\sigma^2}} \quad \textrm{and} \quad \int_{\mathbb R} g(x;\mu,\sigma) dx = 1 \ \forall \ \mu, \, \sigma > 0 \tag{\$}
\end{equation} 


% un'equazione non numerata: \[ .. \]
\[
\begin{gathered} % equazione multilinea
\mbox{a.)}\ \ \sum_{n=2}^{\infty} \frac{\ln(n)}{n^3} (x-5)^n \qquad \qquad 
\mbox{b.)}\ \ \sum_{n=2}^{\infty} \frac{\ln(n)}{n^2} (x-5)^n \\
\mbox{c.)}\ \ \sum_{n=2}^{\infty} \frac{\ln(n)}{n} (x-5)^n
\end{gathered}
\]


Questa è un'equazione in linea: $y = f(x)+\lambda$. In cui uso la mia definizione per il simbolo dell'energia nel centro di massa: $\ecm{91}$ o \ecm{91} vanno entrambi bene perchè le definzione include \verb!\ensuremath!. %

%referenze a equazione o altri elementi:
Gaussina: vedi equazione \ref{eq:sin}  o \eqref{eq:sin} a pagina \pageref{eq:sin}.

\section{Figure}

% ambiente figura che chiediamo di posizionare "qui" (con [h]) e forziamo il posizionamento con [!]

\begin{figure}[h!]
  \centering
  \includegraphics[width=0.5\textwidth]{atlas.pdf}
  \includegraphics[width=0.3\textwidth]{atlas.png} % pdf, png, jpg, eps sono formati supportati.
  \caption{Higgs Boson signal at $m_H = 126\, \textrm{GeV}$. \label{fig:atlas}}
\end{figure}

Così come per le equazioni aggiungo una referenza: i risultati di ATLAS sono mostrati in figura \ref{fig:atlas}.

\section{Bibliografia}

% usiamo bibtex. La bibliografia si può gestire anche con l'ambiente "thebibliography"

% più sotto il comando \bibliography importa il file .bib. Ora posso usare le citazioni in esso definite.

Ecco la citazione per l'articolo di ATLAS: \cite{Aad_2012}.

Per compilare la bibliografia o aggiornarla occorre procedere a step:

\begin{verbatim}
pdflatex mydocument.tex
bibtex mydocument.aux
pdflatex mydocument.tex
pdflatex mydocument.tex
\end{verbatim}


\bibliography{mybib} % occorre avere un file con nome mybib.bib nella cartella di lavoro



%%%%%%%%
% ORA QUALCOSA IN PIU
%%%%%%%&

% forziamo una nuova pagina

\pagebreak


% evitiamo di stampare il numero di pagina:
\thispagestyle{empty}


% minipage crea floating box in cui è possibile impaginare altro materiale

\begin{minipage}[b]{0.5\textwidth}

  \sffamily %cambio font
  \textbf{Università di Pavia}
  
  \vspace{0.1cm}
  
  \textbf{Dipartimento di FISICA}
  
  \vspace{0.4cm}

  \begin{flushright}
  
    Corso di Metodi Informatici della Fisica \\
    A.A. 2023-24

  \end{flushright}

\end{minipage}
\begin{minipage}[b]{0.48\textwidth}
  $\quad$\includegraphics[width=0.8\textwidth]{einstein.jpeg}
\end{minipage}

\vspace{2cm}


\noindent Si certifica che il giorno \today

\begin{center}

\Large

% uso il grassetto con \bf. E importo il contenuto di un altro file name.tex
% tutto ciò che si può fare da command line può diventare facilmente uno script. Ad esempio... uno script che scrive il nome del partecipente in name.tex e compila questo documento genera automaticamente il certificato di partecipazione!
{\bf
  \input{name.tex}
}


\end{center}

\normalsize

\noindent ha partecipato all'entusiasmante corso introduttivo su \LaTeX.

Il corso è stato progettato per fornire una panoramica delle funzionalità di \LaTeX, un linguaggio di markup ampiamente utilizzato per la preparazione di documenti tecnici e scientifici. Durante il corso, il partecipante ha dimostrato competenze in:

\begin{itemize}[label={-}]
\item introduzione ai principi fondamentali di \LaTeX;
\item creazione e formattazione di documenti;
\item utilizzo di comandi di base per la strutturazione del testo;
\item gestione di tabelle e figure;
\item creazione di equazioni e formule matematiche;
\item gestione delle bibliografie.
\end{itemize}



Con l'acquisizione di queste competenze, il partecipante è ben equipaggiato per utilizzare \LaTeX in una vasta gamma di contesti accademici e professionali.

\vspace{1cm}

\flushright

Pavia, \today






\end{document}
